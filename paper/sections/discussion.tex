\section{Discussion}

\begin{comment}
    Here you discuss how your results are: (1) in-line with previous work; and (2) differ
    (expand) on previous work. Also, you can list the limitations of your work (link to future work)

    (a) Which results match previous findings in the literature?
    (b) Which results differ from previous findings, and why
\end{comment}

This study introduces a novel approach to enhancing ethical decision-making within organizations by leveraging AI-simulated deliberations. Our tool facilitates the inclusion of diverse organizational perspectives into the decision-making process. The positive outcomes observed in the treatment group suggest that such simulations can effectively surface and integrate multiple viewpoints, leading to more comprehensive ethical considerations.

While prior research has extensively explored organizational moral disengagement and the challenges of ethical alignment within corporate structures, there has been a lack of computational tools designed to model and address these issues proactively. Our work extends the application of agent-based modeling beyond traditional domains, demonstrating its utility in simulating internal organizational deliberations to preempt ethical misalignments.

Rather than advising users directly, our tool reflects how different professional roles might interpret a dilemma—an approach that proved highly effective in both objective quality metrics and subjective user ratings. To our knowledge, this is the first empirical evaluation of a plural-agent tool in a high-stakes professional context using senior decision-makers.

Our study also highlights the potential of modular, interpretable AI outputs in applied ethics tools. Unlike end-to-end decision models, our summaries serve as intermediate reasoning aids, allowing users to maintain agency while still benefiting from computational support.
