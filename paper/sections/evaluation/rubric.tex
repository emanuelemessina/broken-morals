\subsubsection{Six-Dimensional Rubric}
\label{sec:rubric}

We began by reviewing relevant literature in areas including argumentation theory, ethical decision-making, dialogue systems, and communication studies. From this review, we identified six key dimensions that collectively capture the qualities we aimed to evaluate. Each dimension is supported by prior research and was selected to align with the goals of our study.
The following list contains each dimension in our rubric, the questions an evaluator ought to answer in order to assigning a score to a participant's response, and the reference onto which the dimension is based:

\begin{description}
  \item[Clarity] \textit{Does the response express its ideas clearly? Is the reasoning easy to follow and well-organized?}
    Based on \citep{mctear2005spoken}, which highlights the importance of structure and readability in effective communication.
  \item[Relevance] \textit{Does the response stay focused on the moral dilemma? Are the points made directly connected to the question?}
    Informed by \citet{habernal2016argument}, which shows that content relevance strengthens argument quality.
  \item[Persuasiveness] \textit{Is the argument convincing? Does it use logic, appropriate tone, and structure to support its claims?}
    Drawn from \citet{johnson2006logical}, which emphasize the role of logical and well-structured reasoning in persuasion.
  \item[Concern for Long-Term Consequences] \textit{Does the response consider future or societal effects of the proposed action?}
    Inspired by the Impact Assessment Card \citep{impactassessment2018cscw}, which promotes ethical foresight.
  \item[Practical Usefulness] \textit{Is the proposed solution realistic? Could it work in practice?}
    Based on \citet{bazerman2012judgment}, who stress the importance of actionable and realistic decision-making.
  \item[Awareness of Context] \textit{Does the response take into account the social, cultural, or situational context of the dilemma?}
    Also from the Impact Assessment Card \citep{impactassessment2018cscw}, which encourages attention to contextual factors in ethical reasoning.
\end{description}

Each dimension was scored on a 5-point Likert scale, from 1 ("very low") to 5 ("very high"). The rubric provided the conceptual framework for qualitative evaluation, while the scoring process itself is described in the next section.
