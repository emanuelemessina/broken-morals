\section{Future Work}

\subsection{Overview}

This study establishes a foundational framework for AI-assisted ethical decision-making that opens multiple avenues for systematic expansion and refinement. These opportunities span three primary dimensions: methodological enhancement, architectural diversification, and empirical scaling.

\subsection{Sample and Scope Expansion}

The current prototype's focus on managerial professionals in UK and US markets represents an initial validation phase that naturally extends to broader populations. Expanding participant diversity across geographic regions would illuminate cultural variations in ethical reasoning and AI acceptance patterns. Geographic expansion would complement professional role diversification, extending beyond managerial positions to include C-suite executives, specialized consultants, and senior analysts within strategic decision-making contexts. Professional diversification would enable investigation of role-specific ethical frameworks and decision-making processes. These expanded samples would support more robust statistical analyses and enhance generalizability across organizational contexts.

\subsection{Architectural Innovation}

The ensemble topology employed in this study—where all AI agents interact directly—represents one configuration within a broader space of multi-agent architectures. Alternative topologies offer distinct advantages for different organizational contexts: hierarchical structures that mirror traditional corporate chains of command, bottom-up consensus-building approaches that prioritize grassroots input, and hybrid models that combine centralized oversight with distributed expertise. Hierarchical architectures would enable testing of authority-driven decision processes, while bottom-up structures would illuminate participatory governance models. These architectural variations would provide insights into how organizational structure influences ethical reasoning quality and acceptance.

\subsection{Comparative Framework Development}

Current findings demonstrate multi-agent superiority over control conditions but leave unexplored the comparison with single-agent AI systems. Systematic comparison between vanilla AI responses (single large language model) and our multi-agent ensemble would quantify the specific value added by role-based interaction and deliberation. Single-agent comparisons would establish baseline performance metrics for ethical reasoning tasks. These baseline metrics would enable precise measurement of multi-agent architectural benefits and inform cost-benefit analyses for implementation decisions.

\subsection{Dilemma Set Expansion}

The current study leverages a targeted subset of five dilemmas selected using a Moral Foundations Questionnaire (MFQ) classifier, with an emphasis on economic themes, from the broader Corporate Moral Issues Dataset. While this subset enables focused evaluation, it necessarily limits the diversity of ethical challenges explored. Future work could expand the dilemma set to include a wider range of scenarios from the full dataset, capturing more varied moral tensions and decision contexts. A richer dilemma pool would allow testing the robustness of the multi-agent system across different thematic domains and ethical ambiguities, ultimately strengthening the framework’s generalizability and diagnostic power.

\subsection{Longitudinal and Process Analysis}

The present study adopts a cross-sectional design, capturing only static snapshots of AI-assisted ethical decision-making. Future research could adopt longitudinal designs to observe how users’ ethical reasoning evolves over time with repeated interaction. In parallel, process-oriented analysis—such as tracking deliberation sequences, argument patterns, and time-on-task—would shed light on how decisions are formed, not just what outcomes are reached. These insights would support a deeper understanding of the human-AI collaboration dynamics and help optimize system design for sustained ethical reasoning improvement.

\subsection{Implementation Pathway}

These research directions collectively point toward a comprehensive research program that transforms the current proof-of-concept into a scalable framework for organizational ethical decision support. The systematic expansion across samples, architectures, comparisons, and frameworks would establish both the theoretical foundations and practical guidelines necessary for real-world deployment.